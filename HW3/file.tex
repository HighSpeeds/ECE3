\documentclass[12pt]{article}
\title{ECE 3 HW 3}

\author{Lawrence Liu}
\usepackage{graphicx}
\usepackage{amsmath}
\setlength{\parindent}{0pt}
\usepackage[section]{placeins}
\begin{document}
\maketitle
With $V_4$ as the reference node we have
$$V_2=16V$$
And from KCL, the following equations for $V_1$ and $V_3$
$$\frac{16-V_1}{5}+\frac{V_3-V_1}{7}=1$$
$$0.5=\frac{V_3-V_1}{7}+\frac{V_3}{6}$$
Solving these we get
$$V_1=8.778V$$
$$V_3=5.667V$$
We have that the current flowing into $V_4$ is 
$$1+\frac{V_3}{6}=\boxed{1.945A}$$
From KCL this must be the magnitude of the current flowing across the battery, from $V_4$ to $V_2$\\\\
Now with $V_2$ as the reference node we have\\\\
If $V_2$ is the ground node, then we must have that $V_4=-16V$, and we will from KCL have the following equations for $V_1$ and $V_3$

$$\frac{-V_1}{5}+\frac{V_3-V_1}{7}=1$$
$$0.5=\frac{V_3-V_1}{7}+\frac{V_3+16}{6}$$
Solving these equations we get
$$V_1=-7.222V$$
$$V_3=-10.333V$$
We have that the current flowing into $V_4$ is 
$$1+\frac{V_3+16}{6}=\boxed{1.945A}$$From KCL this must be the magnitude of the current flowing across the battery, from $V_4$ to $V_2$
\end{document}