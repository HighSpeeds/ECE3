\documentclass[12pt]{article}
\title{Impedence and Phasors Tutorial}

\author{Lawrence Liu}
\usepackage{graphicx}
\usepackage{amsmath}
\setlength{\parindent}{0pt}
\usepackage[section]{placeins}
\begin{document}
\maketitle

\section*{Impedence of Circuit compents}

\subsection*{Impedence of a Capacitor}
A capacitor's IV relation is 
$$i=C\frac{dv}{dt}$$
Now let us consider an voltage that takes the form of $v=Ae^{j\omega t}$ where
$j$ is the imaginary unit, ie $j=\sqrt{-1}$, $\omega$ is the frequency, and $A$ is a scaling factor, therefore we have
$$i=C\frac{dv}{dt}=j\omega C Ae^{j\omega t}=j\omega Cv$$
Rearangeing this we get
$$i\frac{1}{j\omega C}=v$$
This reminds us of ohms law! As a matter of fact we can generalize ohms law to AC ciruits, with the
generalized version of resistance being impedence. In this case our impedence is
$$Z_C=\frac{1}{j\omega C}$$
\subsection*{Impedence of a Inductor}
Let us repeat our derivation for the impedence of a capcitor, this time with an Inductor. We have
that the IV relation for a inductor is
$$v=L\frac{di}{dt}$$
now let the current take the form $i=Ae^{j\omega t}$
therefore we have
$$v=j\omega L Ae^{j\omega t}=j\omega L i$$
Therefore we have
$$Z_L=j\omega L$$
\subsection*{Impedence of a Resistor}
The impedence of a resistor is easy, since we have for any $v$,
$$v=iR$$
$$Z_R=R$$
\section*{Phasors}
Let us consider an AC voltage source, with
$$V=A\cos(\omega t+\theta)$$,
adding on a imaginary compent $A j \sin(\omega t +\theta)$ we have
$$A\cos(\omega t+\theta)+A i \sin(\omega t +\theta)=Ae^{j\omega t+j\theta}=Ae^{j\theta}e^{j\omega t}
$$
Let us define the value $Ae^{j\theta}$ to be the phasor, therefore for a AC voltage source
$V=A\cos(\omega t+\theta)$, its corresponding phasor is $\underline{V}=Ae^{j\theta}$.\\
\\
Now lets consider an example of a AC voltage source
$$V=10\sin(\omega t)$$
How would we find the phasor for this source? well first of all from the trig identies, we know
$\cos(\theta-\frac{\pi}{2})=\sin(\theta)$, therefore we can rewrite $V$ as 
$$V=10\cos(\omega t-\frac{\pi}{2})$$

Therefore adding on an imaginary compent $10j\cos(\omega t-\frac{\pi}{2})$ we get
$$10\cos(\omega t-\frac{\pi}{2})+10j\cos(\omega t-\frac{\pi}{2})=10e^{j\omega t-\frac{\pi}{2}j}$$
Therefore the corresponding phasor for this voltage source is $10e^{-j\frac{\pi}{2}}=-10j$

Now what would the corresponding phasor be for the voltage source in the HW?
\end{document}