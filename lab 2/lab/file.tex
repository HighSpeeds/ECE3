\documentclass[12pt]{article}
\title{ECE 3 Lab 2 Lab Report}

\author{Lawrence Liu and Inesh Chakrabarti}
\usepackage{graphicx}
\usepackage{amsmath}
\setlength{\parindent}{0pt}
\usepackage[section]{placeins}
\begin{document}
\maketitle
\section*{100Hz Measurments}
\begin{center}
    \begin{tabular}{|p{0.2\linewidth} | p{0.2\linewidth} |p{0.2\linewidth} |p{0.2\linewidth} |p{0.2\linewidth} |} 
     \hline
     Waveform & Theoretical $V_{RMS}$ & Scope Measurments $V_{RMS}$ & Voltmeter Measurments $V_{RMS}$& Diffrence (\%) \\ [0.5ex] 
     \hline\hline
     Sine & 1.77V & 1.7639V & 1.7645V &  0.034\%\\
     \hline
     Triangle & 1.44V & 1.4325V & 1.4378V & 0.369\%\\
     \hline
     Square & 2.5V & 2.4932V & 2.4823V & -0.439\%\\ 
     \hline
    \end{tabular}
\end{center}
\section*{1KHz Measurments}
\begin{center}
    \begin{tabular}{|p{0.2\linewidth} | p{0.2\linewidth} |p{0.2\linewidth} |p{0.2\linewidth} |p{0.2\linewidth} |} 
     \hline
     Waveform & Theoretical $V_{RMS}$ & Scope Measurments $V_{RMS}$ & Voltmeter Measurments $V_{RMS}$& Diffrence (\%) \\ [0.5ex] 
     \hline\hline
     Sine & 1.77V & 1.7724V & 1.7224V &  -2.821\%\\
     \hline
     Triangle & 1.44V & 1.4431V & 1.3956V & -3.292\%\\
     \hline
     Square & 2.5V & 2.4916V & 2.3067V & -7.42\%\\ 
     \hline
    \end{tabular}
\end{center}
\section*{25KHz Measurments}
\begin{center}
    \begin{tabular}{|p{0.2\linewidth} | p{0.2\linewidth} |p{0.2\linewidth} |p{0.2\linewidth} |p{0.2\linewidth} |} 
     \hline
     Waveform & Theoretical $V_{RMS}$ & Scope Measurments $V_{RMS}$ & Voltmeter Measurments $V_{RMS}$& Diffrence (\%) \\ [0.5ex] 
     \hline\hline
     Sine & 1.77V & 1.7736V & 0.029V &  -98.319\%\\
     \hline
     Triangle & 1.44V & 1.4408V & 0.024V & -98.29\%\\
     \hline
     Square & 2.5V & 2.4952V & 0.041V & -98.34\%\\ 
     \hline
    \end{tabular}
\end{center}

\textbf{What's  your  observation  regarding  the  voltmeter/DMM  reading’s  accuracy  over  different 
frequencies within the same waveform? Can you guess why that's the case? }
The voltmeter became to perform worse at higher frequcies, because the capcitor setup int eh voltmeter and because the voltmeter sampling rate is set to 1s, both of these will act as a low pass filter, and mean that at higher frequcies, less of the actual wave is measured, thereforce causing the readings to become inacurate.
\\\\
\textbf{Does the voltmeter/DMM perform poorer when measuring square or triangular waves over sine 
waves? Can you guess why that's the case? }
The performance is worse with square wave, likely because 
it contains more higher frequices, which the voltmeter cannot measure well.
As discussed above. 
\section*{Square Wave Fourier Analysis}
\begin{center}
    \begin{tabular}{|p{0.3\linewidth} | p{0.3\linewidth} |p{0.3\linewidth} |} 
     \hline
     Nth Harmonic & Measured Values (dB) & Theoretical Values (dB)\\ [0.5ex] 
     \hline\hline
     1 & 0 & 0\\
     \hline
     2 & -86.360 & $-\infty$\\
     \hline
     3 & -9.867 & -9.542\\
     \hline
     4 & -94.256 & $-\infty$\\
     \hline
     5 & -13.9304 & -13.979\\
     \hline
     6 & -83.295 & $-\infty$\\
     \hline
     7 & -19.1605 & -16.90196\\
     \hline
     8 & -84.5246 & $-\infty$\\
     \hline
     9 & -19.1293 & -19.0848\\
     \hline
     10 & -85.7619 & $-\infty$\\
     \hline
    \end{tabular}
\end{center}
\end{document}